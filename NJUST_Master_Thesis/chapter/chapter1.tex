\setcounter{page}{1}
\pagenumbering{arabic}
\chapter{测试}{
	为了规范博士、硕士和硕士专业学位论文的撰写,根据由国家标准局批准颁发的GB7713-87《科学技术报告、学位论文和学术论文的编写格式》,将博士、硕士和硕士专业学位论文的编写格式及有关标准统一规定如下:
	
	学位论文页面设置一律为:上空30mm,下空24mm,左空25mm,右空25mm,对称页边距,页眉20mm,页脚20mm。用A4(297mm×210mm)标准大小的白纸,装订成册后尺寸为(292mm×207mm)。博士学位论文封面用250g天蓝色卡纸,硕士学位论文封面用160g白色卡纸,硕士专业学位论文用160g米黄色卡纸。
	封二、英文封二、声明和学位论文使用声明采用单页印刷,从中文摘要开始采用双面印刷。
	正文中的一级标题(章目)用小3号加粗宋体,段前段后各空18磅,居左;二级标题(条)用4号加粗宋体,段前段后各空12磅,居左;三级标题(款)用小4号加粗宋体,段前段后各空6磅,居左;四级标题(项)同正文用小4号宋体,行距20磅。数字和字母采用Times New Roman体。样本详见附件。
	\section{好的}
	\subsection{真不错}
	摘要是学位论文内容的不加注释和评论的简短陈述,说明研究工作的目的、实验方法、实验结果和最终结论等。应是一篇完整的短文,可以独立使用和引用,摘要中一般不用图表、化学结构式和非公知公用的符号和术语。标题用3号宋体加粗,居中,正文用小4号宋体。
	摘要分中、外两篇,硕士学位论文摘要中文字数400~600个字,博士学位论文摘要中文字数800~1000个字。英文摘要的内容与中文摘要一致,且需合符语法,语句通顺。
	摘要的装订按中、外文顺序进行,置于声明之后,分别由另页开始。详见附件2.6、2.7。
	
	关键词是为了便于文献标引从该学位论文中选取出来用以表示全文主题内容信息款目的单词或术语,一般选取3~8个。其中关键词三个字用四号宋体加粗,其余用小4号宋体。
	关键词写法的例:关键词:专家系统,模糊数学,枪械设计,知识库
	关键词分为中、外文分别附在中、外文摘要的末尾。见附件2.6、2.7。
	
	目次页由学位论文的一、二、三级标题、致谢、参考文献、附录等的序号、名称和页码组成,目次页置于外文摘要后,由另页开始。其中目录两字用3号宋体加粗,一级标题、致谢、参考文献、附录等用4号加粗宋体,其余为小4号宋体。见附件2.8。
	
	为了规范博士、硕士和硕士专业学位论文的撰写,根据由国家标准局批准颁发的GB7713-87《科学技术报告、学位论文和学术论文的编写格式》,将博士、硕士和硕士专业学位论文的编写格式及有关标准统一规定如下:
	
	学位论文页面设置一律为:上空30mm,下空24mm,左空25mm,右空25mm,对称页边距,页眉20mm,页脚20mm。用A4(297mm×210mm)标准大小的白纸,装订成册后尺寸为(292mm×207mm)。博士学位论文封面用250g天蓝色卡纸,硕士学位论文封面用160g白色卡纸,硕士专业学位论文用160g米黄色卡纸。
	封二、英文封二、声明和学位论文使用声明采用单页印刷,从中文摘要开始采用双面印刷。
	正文中的一级标题(章目)用小3号加粗宋体,段前段后各空18磅,居左;二级标题(条)用4号加粗宋体,段前段后各空12磅,居左;三级标题(款)用小4号加粗宋体,段前段后各空6磅,居左;四级标题(项)同正文用小4号宋体,行距20磅。数字和字母采用Times New Roman体。样本详见附件。
	
}
\chapter{验证}{
	为了规范博士、硕士和硕士专业学位论文的撰写,根据由国家标准局批准颁发的GB7713-87《科学技术报告、学位论文和学术论文的编写格式》,将博士、硕士和硕士专业学位论文的编写格式及有关标准统一规定如下:
	
	学位论文页面设置一律为:上空30mm,下空24mm,左空25mm,右空25mm,对称页边距,页眉20mm,页脚20mm。用A4(297mm×210mm)标准大小的白纸,装订成册后尺寸为(292mm×207mm)。博士学位论文封面用250g天蓝色卡纸,硕士学位论文封面用160g白色卡纸,硕士专业学位论文用160g米黄色卡纸。
	封二、英文封二、声明和学位论文使用声明采用单页印刷,从中文摘要开始采用双面印刷。
	正文中的一级标题(章目)用小3号加粗宋体,段前段后各空18磅,居左;二级标题(条)用4号加粗宋体,段前段后各空12磅,居左;三级标题(款)用小4号加粗宋体,段前段后各空6磅,居左;四级标题(项)同正文用小4号宋体,行距20磅。数字和字母采用Times New Roman体。样本详见附件。
	
	摘要是学位论文内容的不加注释和评论的简短陈述,说明研究工作的目的、实验方法、实验结果和最终结论等。应是一篇完整的短文,可以独立使用和引用,摘要中一般不用图表、化学结构式和非公知公用的符号和术语。标题用3号宋体加粗,居中,正文用小4号宋体。
	摘要分中、外两篇,硕士学位论文摘要中文字数400~600个字,博士学位论文摘要中文字数800~1000个字。英文摘要的内容与中文摘要一致,且需合符语法,语句通顺。
	摘要的装订按中、外文顺序进行,置于声明之后,分别由另页开始。详见附件2.6、2.7。
	
	关键词是为了便于文献标引从该学位论文中选取出来用以表示全文主题内容信息款目的单词或术语,一般选取3~8个。其中关键词三个字用四号宋体加粗,其余用小4号宋体。
	关键词写法的例:关键词:专家系统,模糊数学,枪械设计,知识库
	关键词分为中、外文分别附在中、外文摘要的末尾。见附件2.6、2.7。
	
	目次页由学位论文的一、二、三级标题、致谢、参考文献、附录等的序号、名称和页码组成,目次页置于外文摘要后,由另页开始。其中目录两字用3号宋体加粗,一级标题、致谢、参考文献、附录等用4号加粗宋体,其余为小4号宋体。见附件2.8。
	摘要是学位论文内容的不加注释和评论的简短陈述,说明研究工作的目的、实验方法、实验结果和最终结论等。应是一篇完整的短文,可以独立使用和引用,摘要中一般不用图表、化学结构式和非公知公用的符号和术语。标题用3号宋体加粗,居中,正文用小4号宋体。
	摘要分中、外两篇,硕士学位论文摘要中文字数400~600个字,博士学位论文摘要中文字数800~1000个字。英文摘要的内容与中文摘要一致,且需合符语法,语句通顺。
	摘要的装订按中、外文顺序进行,置于声明之后,分别由另页开始。详见附件2.6、2.7。
}